\section*{Introduction}
Before jumping into the theory behind a shared encrypted network file system, it is important to lay a foundation about what a file system is, why it requires sharing capabilities and why security is paramount. A file system (FS) manages and provides access for resources. Generally, resources are composed of folders containing files or other folders. This way, an organizational hierarchy of resources can be established.
    
\lstinputlisting[label={lst:ubunturoot}, caption=Ubuntu root folder, language=bash,style=bashStyle]{./listings/ubunturoot.sh}

Access should be handled so that some roles can access some resources, this is generally done by using dedicated roles such as administrators, owners, guests or even users. An example can be seen in \autoref{lst:ubunturoot}, it shows all folders and files contained within the root directory of a machine running Ubuntu, a popular Linux distribution. The root directory is the top-level directory in the file system's hierarchy. Left of the resource names (\lstinline{boot}, \lstinline{dev}, \lstinline{etc}...) we can find their relative permissions. For instance, the \lstinline{boot} directory is marked as \lstinline{[drwxr-xr-x 4.0K]  boot} which means:

\begin{itemize}
    \item The \lstinline{d} in first position indicates the nature of the resource, in this case a directory.
    \item The pattern \lstinline{rwxr-xr-x} translates to the owner having read, write, and execute permissions, while the group and others have read and execute permissions only.
\end{itemize}

Each triplet (\lstinline{rwx}) relates to a specific permission class. In a Unix-like file system such as the one depicted in \autoref{lst:ubunturoot}, these classes are defined in order as "Owner-Group-Others". This means that for the \lstinline{boot} directory:

\begin{itemize}
    \item The Owner has read (\lstinline{r}), write (\lstinline{w}) and execute (\lstinline{x}) permissions.
    \item The Group that is associated with the directory and all other users only have read and execute permissions.
\end{itemize}

This way any access to the directory is dynamically limited and permissions can be revoked, approved and modified very easily.

\lstinputlisting[label={lst:touch-foo}, caption=Trying to create a file in the root folder without write permissions, language=bash,style=bashStyle]{./listings/touchfoo.sh}

The reason such strategies are put in place is not only related to security. The root directory, such as the one from the Linux distribution, contains all the resources necessary for the proper functioning of the operating system (OS). Modifying these files may cause irreversible configurations that may break the proper flow of the OS and cause failure. The user trying to create a text file using the \lstinline{touch}\cite{touch} command in \autoref{lst:touch-foo} is being denied by the operating system. The current directory being the root directory, one needs so called root-privileges to create, modify or delete any resources\footnote{There are many ways of obtaining some of these privileges notably the \lstinline{sudo} command which will give the user elevated permissions. It is important to note however, that the user trying to \lstinline{sudo} a command must be part of a \lstinline{sudoers} group}. 

To further emphasize the importance of the mechanic surrounding permissions, we must talk about security. A machine such as the one depicted above, may be of use to multiple actors. Therefore, systems must be in place to regulate access across user sessions and resources. Alice may not want Bob to read, edit or delete some of her files or even worse create some in her name. This means that a file system must also be able to obfuscate files from eavesdroppers and make them unavailable, unreadable and uneditable to malicious attackers. This means that in order to consult her own files, Alice must be logged in otherwise her files are inaccessible. But there might be a file which Alice wants to share with Bob for a project, in this case she may create a group, add Bob to this group and using the appropriate triplet from earlier, specify that people of that group may read and write to this file. 

\subsection*{Project goal}
The goal of this project is to design a shared encrypted network file system. The most important aspect of this FS is the manner in which security is guaranteed. In this report, we will outline the general structure of the FS and for each component, specify the necessary cryptographic tools used to provide trust and security.



