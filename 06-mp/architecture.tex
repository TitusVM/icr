\section{Architecture}
\subsection{Web Server}

\begin{minipage}{0.44\textwidth}
    A web server is an online machine that serves clients. Its purpose can be manifold. In our project, this web server will store files for users. They may connect to the server using some credentials (in our case a unique username and a password) and have access to their files as well as the files that have been shared with them by other users. In \autoref{fig:generalarchi} we outlined the general architecture of the server. The server prompts the client for credentials, the client provides them, the server can then verify the credentials and grant access to the storage. In this way, the client can connect from anywhere and on any device, without having to transport the entire storage. This is essentially what is commonly referred to as a cloud storage service.
\end{minipage}
\begin{minipage}{0.47\textwidth}
    \begin{figure}[H]
        \includesvg[inkscapelatex=false, scale=0.80]{resources/general_archi}
        \caption{\label{fig:generalarchi} General architecture of a web server with file saving and sharing capabilities }
    \end{figure}
\end{minipage}


The standard procedure to interact with the server is outlined in \autoref{fig:generalinteract}. As depicted, there are four components that need to be conceptualized.

\begin{minipage}{1\textwidth}
    \begin{figure}[H]
        \centering
        \includesvg[inkscapelatex=false, scale=0.81]{resources/general_interact.svg}
        \caption{\label{fig:generalinteract} General interaction graph }
    \end{figure}
\end{minipage}

\subsection{Storage}

\begin{minipage}{1\textwidth}
    \begin{figure}[H]
        \centering
        \includesvg[inkscapelatex=false]{resources/file_system.svg}
        \caption{\label{fig:filesystem} A basic filesystem }
    \end{figure}
\end{minipage}

The building blocks for a simple file system are as follows:
\begin{itemize}
    \item Folders are containers for other folders and resources.
    \item Resources are bundles of information that manifest in the form of files.
    \item A root folder which is the root node of the hierarchy that illustrates the file system. In \autoref{fig:filesystem} the root folder is the left most folder.
\end{itemize}

A file can therefore be defined as a unique place in the file system by a specific identifier which indicates the path one must travel to find it. To be able to retrieve the files in the green folder in \autoref{fig:filesystem}, one must start to go into the second folder in the root folder and then move to the first folder contained within. So if we translate this to a path, we would have something akin to \lstinline{\root\second_folder\first_folder} which could be defined as the unique identifier for the green folder. We may omit the \lstinline{\root} indicator because all resources are contained within the \lstinline{root} directory and also replace the descriptive names with incremental identification numbers (\lstinline{first_folder} would therefore carry the identifier \lstinline{0}, \lstinline{second_folder} would carry \lstinline{1} and so on). The resulting unique identifier for the green folder would therefore be: \lstinline{10}. Similarly, the blue folder is identified by \lstinline{00}. A file could be described in the same way.
